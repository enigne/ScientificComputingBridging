%\documentclass[12pt]{article}
\documentclass[a4paper,12pt]{article}
\usepackage{amsmath, amssymb, amsthm}
\usepackage{graphicx}
\usepackage{mathtools}
\usepackage{array}

\usepackage{xcolor}
\usepackage[colorlinks = true,
            linkcolor = purple,
            urlcolor  = blue,
            citecolor = blue,
            anchorcolor = blue]{hyperref}


\usepackage{geometry}\geometry{
 a4paper,
 total={210mm,297mm},
 left=20mm,
 right=20mm,
 top=20mm,
 bottom=20mm,
 }
\setlength{\parindent}{0cm}
\renewcommand{\baselinestretch}{1.10}
\pagestyle{plain}

\setlength{\parskip}{0.5em}


\title{Scientific Computing, Bridging Course, 5.0 hp }
\author{1TD045}
\date{\small{Version from September 1, 2019}}
% \author{Cheng Gong}
% \date{Oct 26, 2018}

\begin{document}

\maketitle

\setlength{\leftskip}{0em}
\subsection*{Department in charge of the course}
\setlength{\leftskip}{1em} 

Department of Information Technology\\
Division of Scientific Computing\\
Polacksbacken, Building 2\\
Postal address: Box 337, 751 05 Uppsala\\
Student Office, Room 4204a: Office Hours Mon--Thu 10.00--12.30; Phone 018-4717604

\setlength{\leftskip}{0em}
\subsection*{Teachers}
\setlength{\leftskip}{1em}
%%%%%%%%%%%%%%%%%%%%%%%%%%%%%%%%%%%%%%%%%%%%%%%%%%%%%%%%%%%%%%%%%%%%%%%%%%%%%%%%%%%%%%%%%%%%%%%%%%%%%%%%%%%%%%%%%%%%%%%%%%%%%%%
\begin{itemize}
	\item Cheng Gong, \href{mailto:cheng.gong@it.uu.se}{cheng.gong@it.uu.se}, room ITC 2448
\end{itemize}
%%%%%%%%%%%%%%%%%%%%%%%%%%%%%%%%%%%%%%%%%%%%%%%%%%%%%%%%%%%%%%%%%%%%%%%%%%%%%%%%%%%%%%%%%%%%%%%%%%%%%%%%%%%%%%%%%%%%%%%%%%%%%%%
\setlength{\leftskip}{0em}
\subsection*{Course literature}
\setlength{\leftskip}{1em}
[M]{\label{M}} Michael T. Heath, Scientific Computing⎯An Introductory Survey, Second Edition, McGraw-Hill, International Edition, 2002

[A] Andreas Hellander, Stochastic Simulation and Monte Carlo Methods, Uppsala University, Department of Information Technology, 2009
 
%%%%%%%%%%%%%%%%%%%%%%%%%%%%%%%%%%%%%%%%%%%%%%%%%%%%%%%%%%%%%%%%%%%%%%%%%%%%%%%%%%%%%%%%%%%%%%%%%%%%%%%%%%%%%%%%%%%%%%%%%%%%%%%
\setlength{\leftskip}{0em}
\subsection*{Course homepage}
\setlength{\leftskip}{1em}
All course-related information is collected at \url{http://studentportalen.uu.se}%%%%%%%%%%%%%%%%%%%%%%%%%%%%%%%%%%%%%%%%%%%%%%%%%%%%%%%%%%%%%%%%%%%%%%%%%%%%%%%%%%%%%%%%%%%%%%%%%%%%%%%%%%%%%%%%%%%%%%%%%%%%%%%

\setlength{\leftskip}{0em}
\subsection*{Course outline}
\setlength{\leftskip}{1em}
The course consists of four modules. The theme of each module is indicated below:

\begin{center}
 	\begin{tabular}{ c m{29em}   c m{19em} } 
  	\emph{Module} & \emph{Theme}  \\
 	1 & Introduction to Computer Arithmetic and Matlab \\
	2 & Systems of Linear Equations \\
	3 & Ordinary Differential Equations \\
	4 & Monte Carlo Methods \\
	\end{tabular}
\end{center}

%%%%%%%%%%%%%%%%%%%%%%%%%%%%%%%%%%%%%%%%%%%%%%%%%%%%%%%%%%%%%%%%%%%%%%%%%%%%%%%%%%%%%%%%%%%%%%%%%%%%%%%%%%%%%%%%%%%%%%%%%%%%%%%
\setlength{\leftskip}{0em}
\subsection*{Recommended reading}

\begin{center}
 	\begin{tabular}{ c m{29em}   c m{19em} } 
  	\emph{Module} & \emph{Pages in the course book} \hyperlink{M}{[M]} \\
 	1 & xi--xii, 1--10 (not 1.2.5), 13 (first two paragraphs of 1.2.6), 16 (from 1.2.7)--28, 33 (from 1.4)--39 \\
	2 & 49--79 (not 2.4.8), 84 (from 2.5)--85 (not 2.5.1), 91--92 \\
	3 & 382--404 (not 9.3.5), 405--406, 413--414 \\
	4 & This module is covered by Hellander’s compendium \\
	\end{tabular}
\end{center}

There are Exercises at the end of each chapter in the course book. 
% Solutions are available via the course web site. 
% Please ask the teacher for the username andpassword required to access the solutions. 

%%%%%%%%%%%%%%%%%%%%%%%%%%%%%%%%%%%%%%%%%%%%%%%%%%%%%%%%%%%%%%%%%%%%%%%%%%%%%%%%%%%%%%%%%%%%%%%%%%%%%%%%%%%%%%%%%%%%%%%%%%%%%%%
\setlength{\leftskip}{0em}
\subsection*{Training Activities}
\setlength{\leftskip}{1em}

The course contains three types of training to support your learning: Lab Sessions, Workout Exercises, and Mini Projects.

\subsubsection*{Lab sessions}
The course contains 6 sessions in the computer lab. Each session consists of a
set of computer exercises. They have two objectives: (1) to introduce topics that
will be covered in subsequent lectures, and (2) to introduce elements of Matlab
programming. During the lab session, students will work individually or in
pairs. There will be a teacher present to help out. 

\subsubsection*{Workout exercises}

In order to understand the course contents properly, it is necessary to exercise.
Experience shows that it is useful to do a number of small exercises with pen
and paper to get a good understanding of the details. For this reason, the
students are presented with workout exercises to solve in each module. Students
will work in pairs to solve these exercises, and they will get help from the
teacher when they get stuck. 

\subsubsection*{Mini projects}

The exciting thing about scientific computing is to see how the computational
algorithms are used to address applications where it is necessary to use
computers in order to carry out the computations in reasonable time. The
objective of the mini projects is to let the students experience this.

Before working on each mini project, the teacher will give a brief introduction.
Then the students are expected to work on the project individually. Discussions
among the students and with the teacher are encouraged.


%%%%%%%%%%%%%%%%%%%%%%%%%%%%%%%%%%%%%%%%%%%%%%%%%%%%%%%%%%%%%%%%%%%%%%%%%%%%%%%%%%%%%%%%%%%%%%%%%%%%%%%%%%%%%%%%%%%%%%%%%%%%%%%
\setlength{\leftskip}{0em}
\subsection*{Reports}
\setlength{\leftskip}{1em} 
The educational idea behind the various training activities in the course is that
they will help students to better understand the course contents. An important
aspect of this is that the activities will involve feedback from students to
teachers and vice versa. For this reason it is mandatory for students to report
their progress:
\begin{itemize}
	\item Workout: written solutions should be shown to the teacher.
	\item Mini Project: written reports should be handed in to the teacher.
\end{itemize}


The teacher will inform you about the deadline for each report. It is required
that these deadlines are met. \emph{It is better to leave an incomplete report than to
leave the report too late.}

If you are unable to meet the deadline for special reasons, such as illness, then
you should contact the teacher \emph{before} the deadline.





%%%%%%%%%%%%%%%%%%%%%%%%%%%%%%%%%%%%%%%%%%%%%%%%%%%%%%%%%%%%%%%%%%%%%%%%%%%%%%%%%%%%%%%%%%%%%%%%%%%%%%%%%%%%%%%%%%%%%%%%%%%%%%%
\setlength{\leftskip}{0em}
\subsection*{Workload}
\setlength{\leftskip}{1em}

The course gives 5 credit points. In the Swedish university system, credit points
correspond to expected student workload. Roughly, the student is expected to
work ca. 25--30 hours for each credit point. Consequently, you are expected to
spend ca. 125--150 hours on the present course. The scheduled teaching sessions
amount to ca. 40 hours. This leaves ca. 85--110 hours for your own course work
outside the classroom.

%%%%%%%%%%%%%%%%%%%%%%%%%%%%%%%%%%%%%%%%%%%%%%%%%%%%%%%%%%%%%%%%%%%%%%%%%%%%%%%%%%%%%%%%%%%%%%%%%%%%%%%%%%%%%%%%%%%%%%%%%%%%%%%
\setlength{\leftskip}{0em}
\subsection*{Learning outcomes}
\setlength{\leftskip}{1em}

At the end of the course, you should be able to:

\begin{itemize}
	\item describe the key concepts covered in the course and perform tasks that require knowledge about these concepts;
	\item in general terms explain the ideas behind, and be able to use algorithms
	for solving linear systems, ordinary differential equations and for
	Monte Carlo simulations;
	\item analyze properties of the computational algorithms and mathematical models using the analytical tools presented in the course;
	\item discuss suitable methods and algorithms given an application problem;
	\item given a mathematical model, solve problems in science and engineering	by structuring the problem, choose appropriate numerical method and	generate solution using software and by writing programming code;
	\item present, explain, summarize, evaluate and discuss solution methods and results.
\end{itemize}

%%%%%%%%%%%%%%%%%%%%%%%%%%%%%%%%%%%%%%%%%%%%%%%%%%%%%%%%%%%%%%%%%%%%%%%%%%%%%%%%%%%%%%%%%%%%%%%%%%%%%%%%%%%%%%%%%%%%%%%%%%%%%%%
\setlength{\leftskip}{0em}
\subsection*{Grades and Examination}
\setlength{\leftskip}{1em}


There are two grades for this course: Pass and Fail. The fundamental criterion
for passing the course is to meet the goals stated above. In order to pass the
course, you need to

\begin{enumerate}
	\item solve all the mandatory workout problems correctly,
	\item get all the mini project reports approval by the teacher,
\end{enumerate}

%%%%%%%%%%%%%%%%%%%%%%%%%%%%%%%%%%%%%%%%%%%%%%%%%%%%%%%%%%%%%%%%%%%%%%%%%%%%%%%%%%%%%%%%%%%%%%%%%%%%%%%%%%%%%%%%%%%%%%%%%%%%%%%

\setlength{\leftskip}{0em}
\subsection*{Scholastic Dishonesty}
\setlength{\leftskip}{1em} 
Students may work together and discuss the
homework problems with each other. Copying work done by others is an
act of scholastic dishonesty and will be prosecuted to the full extent
allowed by University policy. For more information on university
policies regarding scholastic dishonesty, see the University of
Uppsala's policy at http://www.it.uu.se/edu/fusk. 

\setlength{\leftskip}{0em}
\subsection*{Students with Disabilities}
\setlength{\leftskip}{1em} 
According to the University
regulation all students with disabilities be guaranteed a learning
environment that provides for reasonable accommodation of their
disabilities. If you need help or want to get more information about
it please contact the University of Uppsala's services for students
with disabilities.  


\end{document}
